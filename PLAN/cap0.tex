\section{COSE DA FARE}
Segue una lista delle cose da fare per ogni assignement da consegnare per il progetto.
Vanno consegnati tutti assieme.

\subsection{Assignment 1}
Creare la vision di un'applicazione mobile che permetta di svolgere:
\begin{itemize}
    \item Tenere Traccia degli oggetti presenti in casa: posizione,caratteristiche, età, etc.
    \item Gestire i capi di abbigliamento posseduti ed i loro possibili abbinamenti: tipo, colore, etc.
    \item Tenere traccia dei prodotti alimentari acquistati: prezzo d'acquisto, quantità, scadenza, etc.
    \item Gestire i turni delle attività tra coinquilini: pulizia, spesa, etc.
    \item Tenere traccia del proprio stato di salute: visite, analisi, assunzione medicinali, etc.
\end{itemize}

Poi bisogna definire le funzionalità che rendano utile l'applicazione allo scopo indicato. 
Bisogna anche cercare per quanto possibile di integrare funzionalità che possano distinguere l'app sul mercato.\\\\

\noindent Applicare le seguenti tecniche:
\begin{itemize}
    \item System concept statement: descrizione testuale dell'idea dell'app
    \item User Personas: definizione della primary persona che costituirà l'utente target dell'app
    \item Requirements brief: lista dei requisiti dell'app con breve descrizione di ogni requisito
    \item Prioritizzazione dei requisiti (in aggiunta alla voce precedente): identificazione del livello di priorità di ogni singolo requisito (utilizzando i modelli visti a lezione o tramite prioritizzazione qualitativa [ALTA, MEDIA, BASSA] in base alla vostra valutazione di quanto la soddisfazione di quel requisito sia essenziale alla user experience dell'app)
\end{itemize}
\noindent Si possono utilizzare tutte le tecniche viste a lezione.
Vanno sempre seguite le indicazioni e le linee guida viste a lezione.
Si deve sfruttare le indicazioni di potenziali utenti target dell'applicazione nel contesto di tutte le tecniche utilizzate. \\ \\

\fcolorbox{red}{white}{\textbf{CONSEGNARE IN FORMATO PDF ENTRO LA DATA DELL'ESAME}}
\newpage

\subsection{Assignment 2}
Creare un prototipo dell'applicazione mobile scelta per lo svolgimento del primo assignement, basandosi sulle seguenti indicazioni.
Tutti gli assignement seguenti sono OBBLIGATORI.

\subsubsection{Sketching}
Usare lo sketching per esplorare velocemente idee di design per diverse schermate dell'applicazione o parti fondamentali delle schermate.
Esplorare possibili design alternativi per quanto riguarda gli aspetti e le funzionalità principali dell'app.
I vari sketch devono essere accompagnati da note che spiegano le schermate e indicano la motivazione per le scelte di design utilizzate, basandosi sulle linee guida e sui pattern visti a lezione.

\subsubsection{Wireframe}
Una volta scelte le idee di design migliori, bisogna creare un wireframe su carta oppure digitalmente utilizzando i tool di prototipazione rapida tipo Marvel. Se vengono effettuate modifiche sostanziali rispetto agli sketch di riferimento, indicare le motivazioni per tali modifiche.

I Wireframe devono essere sottoposti a valutazione con utenti.
La valutazione va documentata (numero di utenti coinvolti, task che gli utenti hanno dovuto svolgere, risultato della valutazione, conseguenze sul design). Siccome la valutazione potrebbe portare a modifiche delle schermate, va tenuta traccia dell'evoluzione del lavoro. 

\subsubsection{OPZIONALI}
Taskflow: Creare un taskflow che documenti le connessioni tra le diverse schermate e come queste avvengono.\\
Prototipo hi-fidelty: utilizzando strumenti di prototipazione come Marvel. Sfruttare se possibile lo strumento delle style tiles per confrontare stili diversi per l'applicazione e scegliere quello più appropriato in funzione del target di utenza.\\

\fcolorbox{red}{white}{\textbf{CONSEGNARE IN FORMATO PDF ENTRO LA DATA DELL'ESAME}}
\newpage